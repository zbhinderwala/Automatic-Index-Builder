\documentclass[24pt]{article}
\usepackage{geometry}
\usepackage{xcolor}
\usepackage{a4}
\usepackage{alltt}
\usepackage{graphicx}
\topmargin=-1cm
\headheight=0in
\textheight=24cm
\textwidth=18cm
\oddsidemargin=-1cm
\evensidemargin=-3cm
\usepackage{epsf}
\usepackage{amsmath}
\usepackage{amssymb}
\usepackage{cite}
\usepackage{hyperref}
%\usepackage{lipsum}
%\usepackage{blindtext}
%\usepackage[toc,page]{appendix}
\DeclareMathOperator{\sech}{sech}
\newcommand{\be}{\begin{equation}}
\newcommand{\ee}{\end{equation}}
\newcommand{\rmnum}[1]{\romannumeral #1}
\newcommand{\Rmnum}[1]{\expandafter\@slowromancap\romannumeral #1@}
\newcommand{\bea}{\begin{eqnarray}}
\newcommand{\eea}{\end{eqnarray}}
\setcounter{MaxMatrixCols}{20} 
\newcommand{\pt}{\partial}
\renewcommand{\baselinestretch}{1.2}

\begin{document}
\title{\bf Causal horizons in a bouncing universe}
\author{Kaushik Bhattacharya, Pritha Bari, Saikat Chakraborty 
\thanks{ E-mail:~ kaushikb, pribari, snilch@iitk.ac.in} 
\\ 
\normalsize
Department of Physics, Indian Institute of Technology,\\ 
\normalsize
Kanpur 208016, India}
\maketitle 
%%%%%%%%%%%%%%%%%%%%
\begin{abstract}
In this article we study the nature of the particle horizon, event
horizon and the Hubble radius in a cosmological model which
accommodates a cosmological bounce. The nature of the horizons and
their variation with time are presented for various models of the
universe. The effective role of the Hubble radius in affecting
causality in such kind of models is briefly discussed.
\end{abstract}
%%%%%%%%%%%%%%%%%%%%%%%%%%%%%%%%%%%%%%%%%%%%%%%%%%%%%%%%%%%%%%%%%%%%%%%%%%%%%%%%%%%%%%%%%%
\section{Introduction}

The issue of causality is at the heart of relativistic physics. In
cosmology, where it is assumed that we live in an expanding universe
modelled on the Friedmann-Lemaitre-Robertson-Walker (FLRW) metric, the
causal nature of the universe is understood by the properties 
of the particle horizon and the event horizon.  In Big-Bang cosmology
it is assumed that there is a finite age of the universe and
intuitively one can visualize that during this time light has
travelled only a finite region and consequently Big-Bang cosmology
predicts a calculable particle horizon for any observer in the present
universe. On the other hand if there was no Big-Bang to start with, as
in non-singular bouncing cosmological
models\cite{Brandenberger:2016vhg,Novello:2008ra}, then it becomes
very difficult to say whether a particle horizon exists for any
observer.  In this article we will show that for non-singular bouncing
cosmologies, where a contracting phase precedes the expansion phase
there may exist certain conditions which dictate when a particle
horizon will exist. Like the particle horizon the event horizon plays
an important role in gravitational theories. In general thermodynamic
behavior of a system can be linked with the event horizon as is done
in black hole physics.  Some authors have even tried to associate
thermodynamic properties of the universe with the cosmic event
horizon\cite{Gibbons:1977mu}.  In the present article we will show
that some of our toy examples of bouncing cosmologies do have event
horizons.

Except the particle and event horizons the Hubble radius also plays an
important role\cite{Lewis:2012yk, Harrison:1991dv} in determining the
causal structure of the universe.  The Hubble radius defines the
Hubble sphere and the surface of the Hubble sphere is called the
Hubble surface.  Many authors use Hubble horizon to describe the
boundary of the Hubble sphere although in this article we will not
associate the word horizon with the Hubble surface.  Later we will see
that in bouncing cosmologies the Hubble surface affects the causal
structure of spacetime in a very subtle way.  A general discussion
illuminating the relationship of the particle horizon and the Hubble
surface, in expanding spacetime, can be found in
Ref.\cite{Davis:2003ad}. The referred article addresses various
misconceptions related to the actual role of the Hubble surface.

Although the concepts of the particle horizon, event horizon and the
Hubble surface are commonly used and well discussed topics in Big-Bang
cosmology\cite{Ellis:2015wdi, Margalef-Bentabol:2012kwa,
  MargalefBentabol:2013bh} they are rarely discussed in the bouncing
universe paradigm where a contracting phase of the universe changes
course and starts to expand again giving rise to the expanding
universe we observe. The bouncing models are interesting because they
do not contain any singularities\cite{Brandenberger:2016vhg,
  Novello:2008ra, Martin:2004pm, Brandenberger:2012zb,
  Cai:2012va}. Once one includes a contracting universe, a universe
which does not have any initial time to start, the concept of the
particle horizon becomes much more involved. Some discussion on the
causality issue in bouncing cosmological scenario is included in
Ref.~\cite{Martin:2003bp}. In a bouncing universe the scale-factor of
the FLRW metric is not a power law function of time during the bounce
and consequently the difference between Hubble radius and the particle
horizon distance (if it exists) diverge maximally near the bounce
point. In this article we will discuss the general nature of particle
horizon, event horizon and the Hubble surface in bouncing cosmological
models. The material in the article is presented in the following way.
In the next section we describe the preliminary theoretical tools
which we will employ throughout the article. In section \ref{pheh} we
quantitatively define the concepts of the horizons and the Hubble
radius.  In section \ref{tbm} we specify two toy bounce models where
all the horizons exist. In the subsequent section \ref{relc} a more
realistic bounce model is discussed. We give multiple examples of this
model using various parameters specifying the bounce.  We discuss
about the results obtained in this article in section \ref{disc} and
finally conclude the paper in the subsequent section.
%%%%%%%%%%%%%%%%%%%%%%%%%%%%%%%%%%%%%%%%%%%%%%%%%%%%%%%%%%%%%%%%%%%%%%%%%%%%%%%%%%%%%%%%%%
\section{Requirements of a cosmological bounce}

In this article we will use the homogeneous and isotropic FLRW
spacetime. We will particularly work with the spatially
flat FLRW metric and its form in the spherical polar coordinates is given by
\begin{eqnarray}
ds^2 = -dt^2 + a^2(t) \left[ dr^2 + r^2(d\theta^2 +
  \sin^2 \theta\,\,d\phi^2)\right]\,,
\label{sfrw}
\end{eqnarray}
where $r$ is comoving radial coordinate. We are using units where the
velocity of light $c=1$. Here $a(t)$ is the scale-factor for the FLRW
spacetime.  The Einstein equation in the cosmological setting can be
expressed in terms of the Hubble parameter $H\equiv \dot{a}/a$, where
a dot over a quantity specifies a time derivative of that quantity.
The relevant equations are
\begin{eqnarray}
H^2 &=&\frac{\kappa}{3}\rho\,,
\label{fried1}\\
\dot{H} &=&-\frac{\kappa}{2}(\rho + P)\,,
\label{fried2}
\end{eqnarray}
where $\rho$ is the energy density and $P$ is the pressure of 
matter which pervades the universe. In the above equations
$\kappa=8\pi G$ where $G=1/M_P^2$, $M_P=1.2 \times 10^{19}\,{\rm
  GeV}$, being the Planck mass. We are using units where $\hbar=1$
where $\hbar$ is the reduced Planck constant. We have assumed that the
cosmological constant term to be zero. The conservation of
energy-momentum leads to another equation
\begin{eqnarray}
\dot{\rho}=-3H(\rho + P)\,.
\label{encon}
\end{eqnarray}
The above set of equations by themselves cannot give proper dynamical
solutions as the number of variables, $a$, $\dot{a}$, $\rho$ and $P$,
exceeds the number of equations. In such a case one assumes a
barotropic equation of state as
\begin{eqnarray}
P=\omega \rho\,,
\label{beos}
\end{eqnarray}
where $\omega$ is supposed to be constant for a particular kind of
matter. The above equations always predicts a spacetime singularity as
$t \to 0$ if the matter content of the universe satisfies some energy
conditions. One can evade the singularity, by introducing a bouncing
universe where the scale-factor remains finite as $t \to 0$, by
violating the null energy condition (NEC) in general relativity.
Although the above set of equation with the specification of a
constant $\omega$ makes the system solvable and one can in principle
find out the scale-factor as a function of time as a solution, in
reality when we apply this theory in the case of a cosmological bounce
the program may get inverted. In this inverted way one assumes a
particular form of the scale-factor to start with. Then using the same
above set of equations one can derive how energy scales with time and
the form of the barotropic ratio. In this approach the barotropic
ratio may not turn out to be a constant in time.  This inverted way of
solving the bouncing problem is applied in various cases because it is
relatively easier to guess a bouncing scale-factor than to guess the
kind of fluid which can produce a cosmological bounce.

If one wants to avoid the Big-Bang singularity near $t=0$ then one can
model a universe where the scale-factor $a(t)$ never becomes zero at
$t=0$. In these models the universe contracts during the time $-\infty
< t \le 0$ and expands during $t \ge 0$ and the cosmological bounce
happens at $t=0$ when $a(t=0) \ne 0$ and $\dot{a}(t=0)=0$. The global
minima of the scale-factor is at the bounce point. The universe
expands after the bounce, implying $\dot{a}$ becomes positive and
increases after the bounce. This second condition implies that
$\ddot{a}(t=0)>0$. The above conditions can also be specified in terms
of the Hubble parameter as:
\begin{eqnarray}
\left. H \right|_{t=0}=0\,,\,\,\,\,\,
\left. \dot{H} \right|_{t=0} > 0\,.
\label{bcond}
\end{eqnarray}
The bounce conditions stated above and Eq.~(\ref{fried2}) immediately
shows that in the flat FLRW spacetime, during bounce
\begin{eqnarray}
\left[\rho + P \right]_{t=0} < 0\,.
\label{nec}
\end{eqnarray}
The above condition specifies that the NEC has to be violated at the
bounce point. The violation of the NEC require exotic
matter\cite{Rubakov:2014jja} in the early universe. One may even
change the gravitational theory to accommodate a cosmological
bounce\cite{Paul:2014cxa, Bhattacharya:2015nda}. In this article we
will not go into the details of the complexities of theories which
generates a cosmological bounce, assuming satisfactory solutions of
these difficult issue exists in principle. We will concentrate on the
main issue of the article which is related to the causality question
in bouncing cosmologies.
%%%%%%%%%%%%%%%%%%%%%%%%%%%%%%%%%%%%%%%%%%%%%%%%%%%%%%%%%%%%%%%%%%%%%%%%%%%%%%%%%
\section{Particle Horizon, Event Horizon and Hubble Radius in bouncing
cosmological models}
\label{pheh}

In the context of, spatially flat, FLRW spacetime let us think of a
light ray which travels from a point $(t_i,R,\theta,\phi)$ to
$(t_0,0,\theta,\phi)$. Light travels along a null geodesic and in our
particular case we have assumed a null, radial geodesic which serves
our purpose. From the line-element we see that for such a null
geodesic
$$ds^2 = -dt^2 + a^2(t) dr^2= 0\,.$$
This above equation gives,
\begin{eqnarray}
\int_{t_i}^{t_0} \frac{dt}{a(t)} = R\,,
\label{tr}
\end{eqnarray}
which gives the comoving distance between the emitter and the
observer. In the above the subscript '$i$' refers to the time of
emission of the light signal from the source, and the subscript '$0$'
refers to the time of reception of the light signal by the
observer.

If the time $t_0$ is the present cosmological time when the observer
is observing the universe, the physical distance to the the emitter in
the observers frame will be $R_P(t_0) = a(t_0) R$.  The regions from
where light could reach the observer at $t=t_0$ forms the region which
may have any causal effect on the present condition of the universe
and regions outside this region can have no causal effect on the
present day universe. All the regions from which light has reached the
present observable universe is enclosed by the causal horizon for the
central observer. The causally connected region is enclosed by a
2-dimensional spacelike spherical surface whose radial physical
coordinate at $t=t_0$ is called $R_P$, and this spacelike surface is
called the particle horizon.  In the bouncing universe, $t_i \to
-\infty$, and we can write,
\begin{eqnarray}
R_P(t_0) \equiv a(t_0)\int_{-\infty}^{t_0} \frac{dt}{a(t)}\,. 
\label{pht}
\end{eqnarray}
We will use the above definition of the particle horizon throughout
this article.

If there exists a finite distance which light can travel in infinite
time in the future, emanating from a spatial point at some time, then
an event horizon can exist. The event horizon is defined by a spatial
two-dimensional spherical surface, whose radius is given by the
finite distance travelled by light in infinite time in the future.
The formal mathematical definition (of the radius) of the event
horizon is:
\begin{eqnarray}
R_E(t_0) \equiv a(t_0)\int_{t_0}^{\infty} \frac{dt}{a(t)}\,,
\label{eht}
\end{eqnarray}
where again $t_0$ is the present time of the observer. 

The Hubble radius is the radial coordinate of the boundary of the
Hubble sphere which is a closed two dimensional spatial surface at any
cosmological time. The Hubble radius is defined as:
\begin{eqnarray}
R_H(t_0) \equiv \frac{1}{|H(t_0)|}\,.
\label{hubs}
\end{eqnarray}
The center of the Hubble sphere is located at the observers
position. The Hubble sphere does not depend upon the past history or
the future of the universe. In the expression of $R_H$ we have
deliberately used the modulus of $H$ to accommodate a contracting
phase of the universe when $H<0$. 

From the definitions of the horizons one can deduce
\begin{eqnarray}
\dot{R}_P &=& 1 + H R_P\,,
\label{rpr}\\
\dot{R}_E &=& -1 + H R_E\,.
\label{rer}
\end{eqnarray}
In the Big-Bang paradigm it is seen that $R_P$ always increases
superluminally as $H R_P$ is positive definite. In the Big-Bang
paradigm the rate of increment of the particle horizon is more than
the expansion rate of the universe. The galaxies on the particle
horizon are receding with a speed $H R_P$ where as the particle
horizon is receding relatively faster and the size of the observable
universe increases with time. In the bouncing scenario more
interesting things can happen.

In the contracting phase of the universe $H<0$ and Eq.~(\ref{rpr})
shows that during this time
$$0 \le \dot{R}_P < 1\,,\,\,\,\,{\rm or}\,\,\,\,\,\,\dot{R}_P<0\,.$$ The
particle horizon distance can increases with time when $H R_P > -1$
and the opposite can happen when $H R_P < -1$. If $HR_P=-1$ then
$\dot{R}_P=0$ and consequently bouncing cosmologies may have a minimum
of the particle horizon distance. When the particle horizon distance
increases with time, during the contraction phase, one must have $|H|
R_P < 1$ or
\begin{eqnarray}
0 \le \dot{R}_P < 1\,,\,\,\,\,\,\,{\rm if}\,\,R_P < R_H\,.
\label{rpin}  
\end{eqnarray}
Similarly, when the particle horizon distance decreases with time,
during the contraction phase, one must have
\begin{eqnarray}
\dot{R}_P <0\,,\,\,\,\,\,\,{\rm if}\,\,R_P > R_H\,.
\label{rpde}  
\end{eqnarray}
The above equations show that during the contraction phase of the
universe the particle horizon can increase subluminally or may
decrease at any rate. On the other hand
\begin{eqnarray}
\dot{R}_P = 0\,,\,\,\,\,\,\,{\rm if}\,\,R_P = R_H\,.
\label{rpc}  
\end{eqnarray}
If the above condition can be maintained for a brief period then
during that period the particle horizon size may remain constant. If
the above condition turns out to be true at only one instant then the
particle horizon will have an extremum at that instant.  From the
above discussion one can predict some properties related to the
particle horizon and the Hubble radius. The points are as follows:
\begin{enumerate}
\item If the particle horizon distance tends to a constant,
  non-singular, value as $t_0 \to -\infty$ then the Hubble radius must
  be equal to the particle horizon distance as $t_0 \to -\infty$.

\item If $R_P(t_0) \to \infty$ as $t_0 \to -\infty$ then in the
  initial phase, or the full phase, of contraction one must have $R_P > R_H$.
\end{enumerate}
One can easily prove the above statements. If $R_P$ tends to a
constant as $t_0$ tends to large negative time then $\dot{R}_P \sim 0$
during that very early period and consequently $R_P \sim R_H$ as $t_0
\to -\infty$. The second statement can be proved by noting the fact
that if $R_P$ is maximum as $t_0 \to -\infty$ then for later times the
particle horizon distance can only decrease. When the particle horizon
distance decreases with time one must have $R_P > R_H$.
 
The way the event horizon grows with time is given in Eq.~(\ref{rer}).
In the contracting phase, where $H R_E$ is negative definite, it can
be easily seen that the event horizon will steeply decrease with time.
All the above points will become apparent in various bouncing models
in the later part of this article.
%%%%%%%%%%%%%%%%%%%%%%%%%%%%%%%%%%%%%%%%%%%%%%%%%%%%%%%%%%%%%%%%%%%%%%%%%%%%%%%%%
\section{The two horizons and the Hubble radius in symmetrical toy
  bounce models}
\label{tbm}

In this section we present some simplified examples where the
functional forms of the scale-factor remains same during $-\infty < t
< \infty$. The bouncing models presented in this section can be
taken as toy examples as in reality the functional form of the
scale-factor of evolution changes its form at different times when the
matter content of the universe undergoes a qualitative transformation,
like the transformation of radiation dominated universe to a universe
filled with dust matter. In the present case we will simply ignore the
various phases of the universe during its evolution and consider only
a bouncing phase which extends in both the temporal directions. In
later sections we will present the more realistic cases. The present
discussion will show how a contracting phase affects the concept of
the horizons discussed in the previous section.

We present two cases, where the scale-factors describing a bouncing
universe, are of the following form:
\begin{eqnarray}
a(t)=\left\{
\begin{array}{ll}
a_0 e^{A t^2}\\
a_0 \cosh (Bt)
\end{array}
\right.
\label{bsfs}
\end{eqnarray}
where $a_0$, $A$ and $B$ are positive real valued
constants\footnote{Here it must be noted that if the spatial sections
  of the FLRW universe is assumed to be closed (spherical slicing)
  then the second scale-factor actually describes a bouncing de Sitter
  universe where the Hubble parameter is a constant, 
  $H=B$. In the present article we are working with flat FLRW metric
  and consequently the second scale-factor does not correspond to a de
  Sitter solution and the Hubble parameter is not a constant.}. The
functional forms of the scale-factors show that both the bounces are
symmetrical in time.  It is to be noted that there can be various
other forms of scale-factors which can give rise to contraction and
bounce. In the later sections of this article we will show some
examples where the evolution of the universe is not symmetrical with
respect to the bounce point.

The particle horizon for the first case is given by
\begin{eqnarray}
R_P(t_0) =  e^{A t_0^2} \int_{-\infty}^{t_0} e^{-A t^2} dt\,,
\end{eqnarray}
assuming the scale-factor remained the same till $t\to -\infty$. The
integral on the right hand side can be evaluated by using the properties of the
Gaussian integrals. After doing the integral one obtains
\begin{eqnarray}
\int_{-\infty}^{t_0} e^{-A t^2} dt=\sqrt{\frac{\pi}{A}}\left[1
  -\frac{1}{2}\,\, {\rm erfc}(\sqrt{A}\, t_0)\right]\,,
\label{phint1}
\end{eqnarray}
where ${\rm erfc}(\sqrt{A}\, t_0)$ is the complimentary error
function.  To see how the above integral behaves in the extreme cases
where $t \to \pm\infty$ one requires the following properties of the
complementary error function,
$$\lim_{x \to -\infty} {\rm erfc}(x)=2\,,\,\,\,\, \lim_{x \to
  \infty} {\rm erfc}(x)=0\,.$$ It must be noted that the above
limits saturates near $x=0$, this information
will help us to figure out the behavior of the particle horizon at
the extreme limits. With all these information we can now write the
expression for particle horizon for the first case as:
\begin{eqnarray}
R_P(t_0) =  e^{A t_0^2} \sqrt{\frac{\pi}{A}}\left[1
-\frac{1}{2}\,\,
{\rm erfc}(\sqrt{A}\, t_0)\right]\,.
\label{case1}
\end{eqnarray}
The limits and their properties of the complimentary error function
listed above shows that the value of the particle horizon remains
finite at both the extremities of the time variable.
%%%%%%%%%%%%%%%%%%%%%%%%%%%%%%%%%%%%%%%%%%%%%%%%%%%%%%%%%%%%%%%%%%
\begin{figure}[t!]
\centering
\includegraphics[scale=.7]{exp_bounce.pdf}
\caption{Plot of Hubble radius represented by continuous blue curve,
  particle horizon radius in orange dashed curve and event horizon
  radius in green dotted curve for $a=e^{t^2}$ bounce model. Here 
  $a_0=1$ sets the scale-factor at the bounce point and $A$ is assumed
  to have unit magnitude. For more explanation see the text below.}
\label{f:etsq}
\end{figure}
%%%%%%%%%%%%%%%%%%%%%%%%%%%%%%%%%%%%%%%%%%%%%%%%%%%%%%%%%%%%%%%%%%


The above expression shows that the particle horizon quickly vanishes
as one goes back in negative time and the particle horizon
increases indefinitely as time evolves, $t \to \infty$. At the bounce
moment, $R_P(0)=\frac12\sqrt{\frac{\pi}{A}}$. The plot of the particle
horizon, represented in orange dashed curve, is shown in Fig.~\ref{f:etsq}. The
above result can be interpreted in a simple way. As the observer moves
back in time, $t \to -\infty$, the observer does not receive any light
from the other parts of the universe from the past as light emitting
particles are infinitely distant from the observer. After a long time
the observer first receives light from its past, and a causal
connection is made. At this moment, $t_0$, the particle horizon
$R_P(t_0)$ comes into account. The amount of light which the observer
receives is coming from the closest sources in the past. As time
proceeds more and more regions from the past are coming in causal
contact with the observer and the process continues forever.

In this case we have $|H|=2A|t|$ and consequently one can easily see
that $R_P(t_0)$ is never proportional to $1/|H(t_0)|$. In general if
the universe admits of both an expanding phase and a contracting phase
it is better to work with $|H|$ instead of $H$ as the Hubble parameter
remains negative during the cosmological contraction. In the present
case we see that near $t\to -\infty$ both $R_P(t_0)$ and $1/|H(t_0)|$
vanishes. At $t=0$, $R_P(0)$ remains finite whereas $1/|H(t_0)| \to
\infty$. The nature of the Hubble surface, plotted in the blue, is
shown in Fig.~\ref{f:etsq}.  The interesting thing to note is the
relative behavior of the Hubble surface and particle horizon before
the bounce. In this toy model the Hubble  radius is always
greater than the particle horizon radius before the bounce and
consequently the particle horizon size increases with time.  Here $R_H
> R_P$ and the two distance scales differ maximally near the bounce
point $t=0$. Although near the bouncing point the universe is moving
towards a momentarily static universe, where dynamical effects of the
expansion can be neglected, $R_H \to \infty$, practically the whole
region inside the 2-surface with radius $R_H$ is causally disconnected
as $R_P$ is finite at $t=0$.

It is interesting to note that this toy example of a bouncing universe
do admit an event horizon. In this case we have
\begin{eqnarray}
R_E(t_0) &=& e^{A t_0^2} \int_{t_0}^{\infty} e^{-A t^2}
dt\nonumber\\ &=& \frac{1}{2}\sqrt{\frac{\pi}{A}}\,\, e^{A t_0^2}
\,\,{\rm erfc}(\sqrt{A}\,\,t_0)\,.
\label{cas1e}
\end{eqnarray}
The above result shows the existence of finite event horizon distance
for a bouncing cosmological model. The behavior of the event horizon
is plotted in green in Fig.~\ref{f:etsq}. The plot shows that the
event horizon diverges at $t_0 \to -\infty$, but in general it has a
finite value for all other times and decreases smoothly as time
evolves. As predicted in the last section, the event horizon steeply
decreases during the contracting phase. In this toy model the event
horizon and particle horizon has the same radius at the bounce point
$t=0$.

In the second case we see that although the scale-factor is very
different when compared to the first case, still the horizon
properties remain qualitatively the same. In this case the particle
horizon is given by:
\begin{eqnarray}
R_P(t_0) = \frac2B \cosh(Bt_0) \tan^{-1}(e^{B t_0})\,.
\label{case2}
\end{eqnarray}
It is seen that the particle horizon asymptotically goes to $1/B$ as
$t_0 \to -\infty$, and $R_P(0)=\pi/2B$ which is finite. The particle
horizon diverges as time increases in the positive direction as
expected. The nature of the particle horizon radius is shown by the
orange dashed curve in Fig.~\ref{f:acosh}.  The interpretation of this
kind of behavior of $R_P(t_0)$ remains the same as before, except the
fact that the particle horizon is non-zero in the far past. In the
present case $|H(t)|=B |\tanh(Bt)|$ and as a result $1/|H(t)|$ tends
to $1/B$ when $t \to \pm \infty$ and it diverges as $t \to 0$. The
Hubble radius is shown by the continuous blue line in
Fig.~\ref{f:acosh}. In this case also we see that $R_P>R_H$ and the
particle horizon size increases with time.  The two length scales
differ maximally near the bounce point as in the last example. In both
the cases we see that at very early times the particle horizon
distance and the Hubble radius becomes equal in magnitude. This
happens as the particle horizon distance attains a constant value
during the very early phase of contraction.
%%%%%%%%%%%%%%%%%%%%%%%%%%%%%%%%%%%%%%%%%%%%%%%%%%%%%%%%%%%%%%%%%%
\begin{figure}[t!]
\centering
\includegraphics[scale=.7]{cosh_bounce.pdf}
\caption{Plot of Hubble radius represented by continuous blue curve,
particle horizon radius in orange dashed curve and event horizon
radius in green dotted curve for $a=\cosh t$ bounce model. Here 
$a_0=1$ sets the scale-factor at the bounce point and $B$ is assumed
to have unit magnitude. For more explanation see the text below.}
\label{f:acosh}
\end{figure}
%%%%%%%%%%%%%%%%%%%%%%%%%%%%%%%%%%%%%%%%%%%%%%%%%%%%%%%%%%%%%%%%%%

Like the previous example, the present bouncing model also admits of
finite event horizon (except at $t \to -\infty$). In the present case,
\begin{eqnarray}
R_E(t_0) =  \frac{2}{B} \cosh
(Bt_0)\tan^{-1}(e^{-B t_0})\,.
\label{case2e}
\end{eqnarray}
The nature of the event horizon radius is shown by the dotted green
curve in Fig.~\ref{f:acosh}. The nature of the event horizons from the
two examples are consistent. Both the plots show that the event
horizon grows indefinitely as one moves back in time and steeply decreases
with time during contraction. 

The nature of the horizons as plotted in Fig.~\ref{f:etsq} and
Fig.~\ref{f:acosh} show an interesting feature. As the bounces are
symmetrical in time the particle horizon and the event horizon are
symmetrical in time. The particle horizon transforms into the event
horizon if one changes the direction of time and vice versa.  In both
the cases discussed above we notice that the particle horizon radius
is an increasing function of time. Once some part of the universe gets
causally connected that part always remains so and more regions of the
universe gets inside the particle horizon as time increases. In the
more realistic bounce model discussed in the next section we will see
that this feature of the particle horizon is lost.
%%%%%%%%%%%%%%%%%%%%%%%%%%%%%%%%%%%%%%%%%%%%%%%%%%%%%%%%%%%%%%%%%%%%%%%%%%%%
\section{A more realistic calculation}
\label{relc}


In this section we assume at least three transformations in the
bouncing universe.  The first transformation occurs during the
contracting phase of the universe. We assume that the contracting
universe was dominated by some form of matter which gave rise to a
scale-factor whose functional form was given by a power law. A power
law scale-factor cannot lead to a cosmological bounce and consequently
we assume that at some time $t^\prime < 0$ the nature of the matter
content/geometry of the universe changed. During $t^{\prime } < t <
t^{\prime \prime}$ where $t^{\prime \prime}>0$, the scale-factor of
the universe changed from the power law form and the new scale-factor
accommodates a bounce at $t=0$. Ultimately the universe comes out of
the bouncing phase at $t^{\prime \prime}$ and the scale-factor again
transforms to a power law function. This is a simple but realistic
description of a cosmological bounce as we know that the scale-factor
of the bouncing universe must have changed to a power law form after
the bounce\footnote{In this article we do not consider the case where
  the bouncing universe culminates in an expanding inflationary
  universe.}. From our understanding of the expanding phase of the
universe and symmetry arguments it is natural to think that some power
law contraction phase may precede the bouncing phase of the
universe. The geometry of the universe changes at $t^\prime$ and
$t^{\prime\prime}$ and in our simplistic model the change happens
instantaneously. In flat FLRW spacetime one can always associate a
series of flat spatial hypersurfaces at each instant of cosmic time
$t$. When the nature of the scale-factor changes at some time $t_0$ it
implies that there is a flat 3-dimensional spatial hypersurface which
separates spacetime into two regions. For $t<t_0$ the scale-factor on
the spatially flat hypersurfaces were given by the old scale-factor,
$a_{\rm old}(t)$, and for spatial slices corresponding to $t>0$ the
scale-factor takes the new value, $a_{\rm new}(t)$.  The surface at
$t_0$ acts like a junction between two temporal regions with different
scale-factors and in general relativity two metrics with different
time dependence can only be matched over a hypersurface if the
junction conditions are satisfied\footnote{A nice discussion on the
  junction conditions in general relativity can be found in
  Ref.~\cite{poisson}.}. For the FLRW metric the junction conditions
state that at the junction, which corresponds to a 3-dim spatial
hypersurface at time $t_0$ in the present case,
$$a_{\rm old}(t_0) = a_{\rm new}(t_0)\,,\,\,\,\,\,\dot{a}_{\rm
  old}(t_0) = \dot{a}_{\rm new}(t_0)\,.$$
When ever the geometry of the universe changes at some time we will assume that
the above junction conditions are satisfied.

We will assume that our model universe has the three following
scale-factors in the three different phases as: 
\begin{eqnarray}
a(t)=\left\{
\begin{array}{ll}
c_0 (-t)^m\,,\,\,\,\,\,\,\,\,t_i \le t < t^\prime\,,\\
a_0 + b_0 t^{2p}\,,\,\,\,\,\,\,\,\, t^{\prime } < t < t^{\prime \prime}\,, \\
d_0 t^n \,,\,\,\,\,\,\,\,\,t^{\prime \prime} < t \le \infty\,.\\
\end{array}
\right.
\label{atc}
\end{eqnarray}
In the above equation $t_i(<0)$ is assumed to be some initial time
which will ultimately tend to $-\infty$. The quantities $m$ and $n$
are positive real constants and $m$ is in general not equal to $n$.
The constant $p$ takes positive integer values.  Out of the three
constants we will assume that $0< n < 1$ as because in an expanding
flat FLRW model this constraint is generally followed. This condition
on $n$ predicts that the bouncing universe scenario we are discussing
at present does not admit an event horizon as it is well known that
the flat FLRW metric does not have any event horizon if the
scale-factor has a power law form where $n$ satisfies the above
constraint. The coefficients $c_0$, $a_0$, $b_0$ and $d_0$ are also
positive real constants out of which $a_0$ normalizes the scale-factor
at the bounce time.  The Hubble radius during the three phases are:
\begin{eqnarray}
R_H(t_0) \equiv \frac{1}{|H(t_0)|}=\left\{
\begin{array}{ll}
\frac{|t_0|}{m}\,,\,\,\,\,\,\,\,\,t_i \le t_0 < t^\prime\,,\\
\left|\frac{a_0 + b_0 t_0^{2p}}{2b_0 p t_0^{2p-1}}\right|\,,\,\,\,\,\,\,\,\,
t^{\prime }
< t_0 < t^{\prime \prime}\, \\
\frac{t_0}{n} \,,\,\,\,\,\,\,\,\,t^{\prime \prime} < t_0 \le \infty\,\\
\end{array}
\right.
\end{eqnarray}
Because of the junction conditions, on the metric, at $t^\prime$ and
$t^{\prime\prime}$ one can easily verify that $R_H(t_0)$ changes
continuously at the junctions.  The particle horizon at any time
during expansion can be written as:
\begin{eqnarray}
R_P(t_0)= d_0 t_0^n \left [\int_{t_i}^{t'} \frac{dt}{c_0(-t)^m}+
  \int_{t'}^{t''} \frac{dt}{a_0 + b_0 t^{2p}} + \int_{t''}^{t_0}
  \frac{dt}{d_0t^n}\right]\,.\,\,\,\,\,\,\,\,(t^{\prime \prime} < t_0
\le \infty)
\label{rpd}
\end{eqnarray}
Unlike the case with the toy models of bounce in the present case we
have to specify the particle horizons in all the three phases
of development of the universe as the scale-factors change during
these phases. The particle horizon radius during the power law
contraction phase is given by
\begin{eqnarray}
  R_P(t_0) = \frac{1}{1-m}[(-t_0)^m (-t_i)^{1-m}+t_0]\,.\,\,\,\,\,\,\,\,
  (t_i \le t_0 < t^\prime)
\label{rpi}
\end{eqnarray}
Similarly, the particle horizon radius during the bouncing phase is given by
\begin{eqnarray}
R_P(t_0) = (a_0 +  b_0t_0^{2p})\left[\frac{1}{c_0(1-m)}\Big((-t_i)^{1-m}-(-t')^{1-m}
    \Big) +\int ^{t_0} _{t'} \frac{dt}{a_0 +  b_0t^{2p}}
  \right]\,.\,\,\,\,\,\,\,\, (t^{\prime } < t_0 < t^{\prime \prime})
\label{rpm}
\end{eqnarray}
The expression of the particle horizon distance during the expansion
phase is
\begin{eqnarray}
R_P(t_0)&=&d_0 t_0^n \left[\frac{1}{c_0(1-m)}\Big((-t_i)^{1-m}-(-t')^{1-m} \Big)
  +\int ^{t''} _{t'} \frac{dt}{a_0 +  b_0t^{2p}}\right.\nonumber\\
&+&\left.\frac{1}{d_0(1-n)}\Big(t_0^{1-n}-t''^{1-n}\Big)\right] \,.
\,\,\,\,\,\,\,\,(t^{\prime \prime} < t_0 \le \infty)
\label{rpf}
\end{eqnarray}
We will specify the the integral containing the term $1/(a_0 + b_0
t^{2p})$ later, at present we concentrate on the junction
conditions. Applying the junction conditions at $t'$ we get
\begin{eqnarray}
  a_0=\left(\frac{2p}{m}-1\right)b_0 t'^{2p}\,,\,\,\,\,\,\,\,\,\,\,
  c_0 (-t')^m=\frac{a_0}{1-(m/2p)}\,.
\label{jptp}
\end{eqnarray}
Similarly applying the junction conditions at $t^{\prime\prime}$ one gets,
\begin{eqnarray}
a_0= \left(\frac{2p}{n}-1\right)b_0 t''^{2p}\,,\,\,\,\,\,\,\,\,\,\,
d_0 t''^n=\frac{a_0}{1-(n/2p)}\,.
\label{jctpp}
\end{eqnarray}
Comparing the above conditions one easily gets
\begin{equation}
\Big (\frac{t''}{t'}\Big)^{2p}  = \frac{\frac{2p}{m}-1}{\frac{2p}{n}-1}\,,
\label{tptpp}
\end{equation}
which sets a relationship between the matching times and the parameters
appearing in the scale-factors. Henceforth whenever we specify the
expression of the particle horizon and the Hubble radius we will
assume that the constants appearing in those expressions satisfy the
above junction conditions. 

The expressions of the particle horizon distances, as given in
Eqs.~(\ref{rpi}), (\ref{rpm}) and (\ref{rpf}), shows that $R_P(t_0)$
for all the phases is finite (as $t_i \to -\infty$) only when
$m>1$. Consequently, the existence of the particle horizon in the
realistic case depends upon the value of $m$.
%%%%%%%%%%%%%%%%%%%%%%%%%%%%%%%%%%%%%%%%%%%%%%%%%%
\subsection{Nature of particle horizon}

In this subsection we will consider $m>1$ and assume $t_i \to -\infty$.
In this article we will present the results for two values of
$p$. In the first case $p=1$ and in the second case $p=2$ both of
which gives rise to a symmetric bouncing phase.
%%%%%%%%%%%%%%%%%%%%%%%%%%%%%%%%%
\subsubsection{The case where $p=1$}

When $p=1$ one can write the expressions for particle horizon distance as:
\begin{eqnarray}
R_P(t_0)=\left\{
\begin{array}{ll}
\frac{t_0}{1-m}\,,\,\,\,\,\,\,\,\,(-\infty \le t_0 < t^\prime)\\
(a_0 +  b_0t_0^2)\left[\frac{1}{c_0(m-1)}(-t')^{1-m} - \frac{1}{\sqrt{a_0
      b_0}} \arctan(\sqrt{\frac{b_0}{a_0}} t')+ \frac{1}{\sqrt{a_0
      b_0}} \arctan(\sqrt{\frac{b_0}{a_0}} t_0)\right]\,,
\,\,\,\,\,\,\,\,(t^{\prime } < t_0 < t^{\prime \prime}) \\  
d_0 t_0^n \left[\frac{1}{c_0(m-1)}(-t')^{1-m} - \frac{1}{\sqrt{a_0
      b_0}} \arctan\Big(\sqrt{\frac{b_0}{a_0}} t'\Big) + \frac{1}{\sqrt{a_0
      b_0}} \arctan\Big(\sqrt{\frac{b_0}{a_0}} t''\Big)\right.\\
\left.+\frac{1}{d_0(1-n)}\
\Big(t_0^{1-n}-t''^{1-n}\Big)\right]\,.\,\,\,\,\,\,\,\,\,(t^{\prime
  \prime} < t_0 \le \infty)
\end{array}
\right.
\end{eqnarray}
Finally we have to choose the constants appearing in the above
expressions judiciously such that the junction conditions are
satisfied. We take $a_0=1$ and $n=1/2$ assuming a radiation dominated
universe just after the bouncing phase. The time instants where the
scale-factors change are assumed to be $t'=-2$ and $t''=1$ in some
units. From the junction conditions one can now easily obtain 
$$m=\frac{8}{7}\,,\,\,b_0=\frac{1}{3}\,,\,\,c_0=\frac{7}{3}2^{-\frac{8}{7}}\,,\,\,
d_0=\frac{4}{3}\,.$$ Using these values we can write the particle
horizon radius at any phase of evolution of the universe. Particle
horizon distance at any time during power law contraction is simply
given by
\begin{eqnarray}
R_P(t_0)= -7t_0\,.
\end{eqnarray}
In this phase $t_0<0$. Particle horizon distance in the intermediate bouncing
phase comes out to be:
\begin{eqnarray}
R_P(t_0)= \Big(1 +  \frac{t_0^2}{3}\Big)\left[6 + \sqrt{3}
  \Big(\arctan\Big(\frac{t_0}{\sqrt{3}}\Big) + \arctan \Big(\frac{2}{\sqrt{3}}\Big)\Big)\right]\,.
\end{eqnarray}
Finally the particle horizon radius during the expanding phase of the
universe is:
\begin{eqnarray}
R_P(t_0)= \frac{4}{3} t_0^{\frac{1}{2}} \left[6 + \sqrt{3} \Big(\arctan{\Big(\frac{1}
    {\sqrt{3}}\Big)} +
  \arctan{\Big(\frac{2}{\sqrt{3}}\Big)}\Big)+\frac32(t_0^{1/2} -1)\right]\,.
\end{eqnarray}
%%%%%%%%%%%%%%%%%%%%%%%%%%%%%%%%%%%%%%%%%%%%%%%%%%%%%%%%%
\begin{figure}[t!]
\begin{minipage}[b]{0.5\linewidth}
\centering
\includegraphics[scale=.7]{bouncing_univ_1.pdf}
\caption{Variation of particle horizon radius in orange dashed line
  and Hubble radius in blue through bounce for p=1.}
\label{f:p1}
\end{minipage}
\hspace{0.2cm}
\begin{minipage}[b]{0.5\linewidth}
\centering
\includegraphics[scale=.7]{bouncing_univ_2.pdf}
\caption{Variation of particle horizon radius in orange dashed line
  and Hubble radius in blue through bounce for p=2.}
\label{f:p2}
\end{minipage}
\end{figure}
%%%%%%%%%%%%%%%%%%%%%%%%%%%%%%%%%%%%%%%%%%%%%%%%%%%%%%%%%%%%%%%%%%%
The particle horizon and the Hubble radius are plotted for the case
$p=1$ in Fig.~\ref{f:p1}. In the plot the Hubble radius is drawn in
blue and the particle horizon is shown by orange dashed curve.  The
plot shows that most of the time the Hubble sphere is causally
connected except very near to the bounce point where the Hubble radius
diverge. We have plotted the behavior of the particle horizon distance
and the Hubble radius near the bounce point and consequently the plots
do not convey the complete information about these distance scales
away from the bounce point. As one goes back in time the Hubble radius
increases and so does the particle horizon distance. The important
feature which comes out of the plot is that the particle horizon
decreases initially and then it attains a minimum value near the
bounce and increases in the expanding phase.
%%%%%%%%%%%%%%%%%%%%%%%%%%%%%%%%%%%%%%%%%%%%%%%%%%%%%%%%%%%%%%%%%%%%%%%%%%%%%%%%%
\subsubsection{The case where $p=2$}

In the present case the integral involving the term $1/(a_0 + b_0 t^{2p})$
yields\cite{grad}
$$\int\frac{dt}{a_0+b_0 t^4}=\frac{\alpha}{4 \sqrt{2}a_0}\left[\ln
  \Big(\frac{t^2 + \sqrt{2} \alpha t + \alpha^2}{t^2 - \sqrt{2} \alpha
    t + \alpha^2}\Big)+2 \arctan \frac{\sqrt{2}\alpha t}{\alpha^2 -
    t^2}\right]\,,$$ where $\alpha=(\frac{a_0}{b_0})^{\frac{1}{4}}$.
Using the above result we can write the particle horizon distance in
the three cases as:
\begin{eqnarray}
R_P(t_0)=\left\{
\begin{array}{ll}
\frac{t_0}{1-m}\,,\,\,\,\,\,\,\,\,(-\infty \le t < t^\prime)\\
(a_0 +  b_0 t_0^4)\left[\frac{1}{c_0(m-1)}(-t')^{1-m} +\frac{ \alpha}{
  4 \sqrt{2}a_0} \Big(\ln \Big[\frac{t'^2 - \sqrt{2} \alpha t' + \alpha^2}{
      t'^2 + \sqrt{2} \alpha t' + \alpha^2} \times \frac{
      t_0^2 + \sqrt{2} \alpha t + \alpha^2}{
      t_0^2 - \sqrt{2} \alpha t + \alpha^2}\Big]\right.\\
   \left. + 2 \arctan\frac{\sqrt{2} \alpha t_0}{\alpha^2 - t_0^2} - 
   2 \arctan\frac{\sqrt{2} \alpha t'}{\alpha^2 - t'^2}\Big)\right]\,,\,\,\,\,\,\,\,\,
(t^{\prime } < t_0 < t^{\prime \prime})\\
d_0 t_0^n \left[\frac{1}{c_0(m-1)}(-t')^{1-m}  +\frac{ \alpha}{
  4 \sqrt{2}a_0} \Big(\ln\Big[\frac{t'^2 - \sqrt{2} \alpha t' + \alpha^2}{
      t'^2 + \sqrt{2} \alpha t' + \alpha^2} \times \frac{
      t''^2 + \sqrt{2} \alpha t'' + \alpha^2}{
      t''^2 - \sqrt{2} \alpha t'' + \alpha^2}\Big]\right. \\
\left. + 2 \arctan\frac{\sqrt{2} \alpha t''}{\alpha^2 - t''^2} - 
2 \arctan\frac{\sqrt{2} \alpha t'}{\alpha^2 - t'^2}\Big)+
\frac{1}{d_0(1-n)}(t_0^{1-n}-t''^{1-n})
\right]\,.\,\,\,\,\,\,\,\,(t^{\prime \prime} < t_0 \le \infty)\\
\end{array}
\right.
\end{eqnarray}
In the present we set $a_0=1$ and $n=1/2$ as done in the previous
case.  The time instants where the scale-factors change are assumed to
be the same as in the previous case. The junction conditions now
predict
$$m=\frac{64}{23}\,,\,\,b_0=\frac{1}{7}\,,\,\,c_0=\frac{23}{7}2^{-\frac{64}{23}}\,,\,\,
d_0=\frac{8}{7}\,.$$ The resulting particle horizon distance is
plotted in orange dashed curve in Fig.~\ref{f:p2}. The Hubble radius
at each instant is plotted in blue curve. The present plot is
qualitatively same as the one in Fig.~\ref{f:p1}. The only difference
between them is that for the case $p=2$ the radiation dominated
universe can have a certain region where the Hubble radius exceeds
the particle horizon distance.  Both the curves show that the particle
horizon follows a smooth curve which has a minima near the bounce
point. 

In both the above cases we observe that during the contraction phase
the Hubble surface lies within the particle horizon. From our
discussion in section \ref{pheh} one can infer that in such cases the
particle horizon size must decrease with time. This behavior of the
particle horizon is in contrast to the corresponding behavior of the
particle horizons in the toy model examples.
%%%%%%%%%%%%%%%%%%%%%%%%%%%%%%%%%%%%%%%%%%%%%%%%%%%%%%%%%%%%%%%%%%%%%%%%%%%%%
\section{Discussion}
\label{disc}

We discuss the important points regarding horizons in bouncing
cosmologies in this section. At first we will like to point out the
the role of various energy conditions in deciding the fate of the
existence of the horizons. We start with the toy examples.
In the first case in the toy examples of bounce one can easily show that
$$\rho=\frac{12A^2}{\kappa}t^2 \ge 0\,,\,\,\,\,\rho+P = -\frac{4A}{\kappa}
<0\,,$$ when $A>0$. In the second case we have
$$\rho=\frac{3B^2}{\kappa}\tanh^2(Bt) \ge 0\,,\,\,\,\,
\rho+P=-\frac{2B^2}{\kappa}\sech^2 (Bt) < 0\,.$$ Both the above
examples show that in these cases the strong energy condition (SEC),
the weak energy condition (WEC) and the NEC are violated during
$-\infty < t < \infty$.  In these cases all the horizons exist.

In the realistic case of cosmological bounce presented in the paper we
see that if the contracting phase prior to the bouncing phase has a
power law scale-factor then the particle horizon exists only when
$m>1$. As in general relativity the exponent in the power law is
related with the barotropic equation of state $\omega$ via
$$\omega=\frac{2-3m}{3m}\,,$$ it is seen that $\omega \ge 0$ only
when $m \le 2/3$, where the equation of state becomes zero at
$m=2/3$. If $m > 2/3$ the barotropic ratio becomes negative. As a
consequence it follows that if there is a power law contraction phase
before the bouncing phase then the condition $\omega<0$ ensures the
existence of a finite particle horizon. The condition that $m>1$
translates to
\begin{eqnarray}
-1< \omega < -\frac13\,,  
\label{secv}    
\end{eqnarray}  
which specifies that in such a case $\rho + 3P < 0$ during the
contraction phase. This result shows that the particle horizon in such
cases can only exist if the SEC is violated during the power law
contraction phase. On the other hand $\rho + P>0$ and the WEC will
hold during the contraction phase. Near the bounce point all the above
energy conditions will be violated.

In the realistic bounce model we have used various forms of the
scale-factors in the various evolutionary phases of the universe. The
different metric smoothly transforms from one form to the other
because of the junction conditions. The junction conditions and our
choice of the bouncing scale-factor combines to produce an interesting
effect. It is apparent from Eq.~(\ref{tptpp}) that if one chooses
$t^{\prime\prime}=1$, $t^\prime=-1$ and $n=1/2$ then $m$ also turns
out to be $1/2$ when $p$ is an integer. The junction condition and
symmetric matching times combine to predict a symmetrical evolution of
the universe through bounce. If we want to have an asymmetrical
evolution of the universe the matching times should be different which
will lead to dissimilar values of $m$ and $n$ or one may choose to
have $m \ne n$ which will give asymmetric matching times.

In the toy examples we saw the minima of the particle horizons appear
near $t \to -\infty$. The particle horizons then grows
monotonically. The particle horizon radius displays such a behavior
because in both the toy examples the scale-factors at $t \to -\infty$
diverge too fast as one moves back in time and consequently
considerable amount of light cannot reach any region of the universe
in the first case. In the second case only a small region remains
causally connected as $t \to -\infty$. In these cases, the surface
defining the particle horizon, at a particular time, have a subluminal
velocity of contraction. As a consequence, centrally directed,
in-coming, light from outside the particle horizon can overtake this
surface and reach the central observer. In this case the particle
horizon increases as more light from distant parts of the universe
reach the central observer as time increases. In the realistic cases,
where particle horizon exists, we observe that the minima of the
particle horizon distance is always near the bounce point. In these
cases the scale-factor during the contraction phase is given by a
power law function of time which diverges as $t \to -\infty$ but this
divergence is much milder than the divergences of the scale-factors in
the toy models. As one moves back in time a much wider part of the
universe seems to be causally connected in the realistic cases.  But
as in these cases the particle horizon distance exceeds the Hubble
radius the surface defining the particle horizon, at a particular
time, is radially moving inward in a superluminal way. Photons which
are moving radially inward from outside this surface will not be able
to reach this surface.  As the surface contracts with time the
particle horizon distance diminishes with time. It is to be noted that
a contracting particle horizon does not imply that causal regions of
the universe are moving out of the horizon, it implies that as the
universe contracts no new causal regions are entering the particle
horizon. 

The minima of the particle horizon appears near the bounce point in
the more realistic models. This happens when $R_P=R_H$. In the
examples given in this paper the condition $R_P=R_H$ happens at an
instant and consequently the particle horizon attains a minima at that
instant. If the two surfaces, describing the particle horizon and the
Hubble surface, remain identical for some period of time then during
that period the particle horizon distance will remain constant. This
will happen because light cannot enter radially inward into the
particle horizon as the surface defining the particle horizon, at a
particular time, contracts with the speed of light. More over as the
spatial surface defining the particle horizon, at a particular time,
is moving inwards with just the velocity of light all the emitters
inside this surface move inwards subluminally and the light they emit
can reach the observer in due time. In this case neither the particle
horizon distance increases nor it decreases with time.

It must be noted that the surface, with definite physical radius,
which define the particle horizon or the Hubble surface at a
particular time, $t$, may not define the particle horizon or the
Hubble surface at a later time $t^\prime>t$. Another surface, with a
different physical radius, will define the particle horizon at
$t^\prime$.  As an example, the surface which coincides with the
particle horizon at time $t$, in the case where $R_P < R_H$, contracts
with time as its physical radial coordinate shrinks with time.  On the
other hand the particle horizon distance increases with time. The
surface which defines the Hubble sphere at any particular time, $t$,
will contract in time where as the Hubble radius may increase with
time.

Although in the Big-Bang paradigm the particle horizon distance and
the Hubble radius are proportional to each other when the scale-factor
of expansion is given by a power law function in bouncing models this
fact does not hold anymore. In the bouncing models the power law
expansion phase may accommodate a particle horizon and the
relationship between the particle horizon distance and the Hubble
radius is much more complex.  Although the present authors have not
seen any article addressing the issue related to the particle horizons
in bouncing models, a recent publication \cite{Barrau:2017ukm} discuss
the effect of bouncing models on luminosity distance in cosmology.

If a bouncing universe does not have a finite particle horizon then
the only compact surface which determines the causal structure of the
universe is the Hubble surface. Although the Hubble radius diverges
during the bounce time, it is the only compact 2-dimensional surface
which can have any say on the causal structure of the universe during
the contracting and expanding phases.

The above points encapsulate the important conclusions drawn from the
present work. The above points specify the complex nature of the
causality problem encountered in cosmological models which accommodate
a cosmological bounce. It is evident from the discussion that there
can be various conditions dictating the presence of the particle
horizon and the event horizon in bouncing cosmologies. In some cases
both may cease to exist. If the contraction phase is dominated by dust
and the expansion phase dominated by radiation non of the horizons
exist. In some cases only the event horizon may exist.  If the
contracting phase is dust dominated and the future expanding universe
is de Sitter then we expect that event horizon will exist even though
the particle horizon does not exist. In some, artificial cases, both
may exist as in the case of the toy examples presented in the article.
%%%%%%%%%%%%%%%%%%%%%%%%%%%%%%%%%%%%%%%%%%%%%%%%%%%%%%%%%%%%
\section{Conclusion}

The present article addresses the topic related to causality in
general bouncing cosmological models based on the general relativistic
flat FLRW solution.  Keeping the standard definitions of the particle
horizon, event horizon and Hubble radius the present article
generalizes their meaning in a bouncing universe which accommodates an
infinitely stretched (in time) contraction phase.  It is shown that in
many bouncing cases the particle horizon may not exist. On the other
hand if matter content of the universe, during the contraction phase,
violates some energy energy conditions then the particle horizon can
exist in some simple models. The realistic models studied in the paper
are simple as they involve a single phase change, when the
scale-factor changes, during the contraction period. One can model
more complex contraction phases in the future but the existence of the
particle horizon will mainly depend upon the earliest phase one
considers during contraction.

Two toy model bounce scenario is presented at first which illuminates
the generalization of the concepts of the horizons in the bouncing
scenario. The toy models have a single scale-factor during
contraction, bounce and expansion phases. In the examples we show 
the toy models have both the horizons. 

A more realistic model of cosmological bounce, which accommodates
three phases of evolution of the universe, is presented in the
article. The realistic case exposes the difficulty in calculating the
particle horizon distance as the properties of particle horizon
becomes dependent on the earliest history of the bouncing models.  In
the more realistic calculation, presented in the article, the
contraction and the expansion phases are guided by a power law
scale-factor where as the bouncing scale-factor is assumed to be an
even function of time. It is shown that in the more realistic models of bounce 
there can be some cases where the particle horizon may
exist. The criterion for existence of particle horizons depends upon
the energy conditions followed by matter during contraction. We have
emphasized the special role of the Hubble surface which affects
the time evolution of the particle horizon.

The present work shows that the causality problem in bouncing universe
is intrinsically related to an understanding of the various phases of
the universe during the contraction phase. As our understanding of the
contraction phase is purely speculative at present the models we use
to figure out the nature of particle horizon remains over
simplistic. The present authors believe that although the causality
problem in bouncing universe models are far from being solved the
present article shows the qualitative and quantitative difficulties
one must have to circumvent in the future to produce more meaningful 
results.
%%%%%%%%%%%%%%%%%%%%%%%%%%%%%%
\begin{thebibliography}{999}

%\cite{Brandenberger:2016vhg}
\bibitem{Brandenberger:2016vhg} 
  R.~Brandenberger and P.~Peter,
  %``Bouncing Cosmologies: Progress and Problems,''
  Found.\ Phys.\  {\bf 47}, no. 6, 797 (2017)
  doi:10.1007/s10701-016-0057-0
  [arXiv:1603.05834 [hep-th]].
  %%CITATION = doi:10.1007/s10701-016-0057-0;%%
  %64 citations counted in INSPIRE as of 24 Nov 2017

%\cite{Novello:2008ra}
\bibitem{Novello:2008ra} 
  M.~Novello and S.~E.~P.~Bergliaffa,
  %``Bouncing Cosmologies,''
  Phys.\ Rept.\  {\bf 463}, 127 (2008)
  doi:10.1016/j.physrep.2008.04.006
  [arXiv:0802.1634 [astro-ph]].
  %%CITATION = doi:10.1016/j.physrep.2008.04.006;%%
  %368 citations counted in INSPIRE as of 23 Nov 2017 

%\cite{Gibbons:1977mu}
\bibitem{Gibbons:1977mu} 
  G.~W.~Gibbons and S.~W.~Hawking,
  %``Cosmological Event Horizons, Thermodynamics, and Particle Creation,''
  Phys.\ Rev.\ D {\bf 15}, 2738 (1977).
  doi:10.1103/PhysRevD.15.2738
  %%CITATION = doi:10.1103/PhysRevD.15.2738;%%
  %1924 citations counted in INSPIRE as of 24 Nov 2017

%\cite{Lewis:2012yk}
\bibitem{Lewis:2012yk} 
  G.~F.~Lewis and P.~van Oirschot,
  %``How does the Hubble Sphere limit our view of the Universe?,''
  Mon.\ Not.\ Roy.\ Astron.\ Soc.\  {\bf 423}, L26 (2012)
  doi:10.1111/j.1745-3933.2012.01249.x
  [arXiv:1203.0032 [astro-ph.CO]].
  %%CITATION = doi:10.1111/j.1745-3933.2012.01249.x;%%
  %9 citations counted in INSPIRE as of 29 Nov 2017

%\cite{Harrison:1991dv}
\bibitem{Harrison:1991dv} 
  E.~R.~Harrison,
  %``Hubble spheres and particle horizons,''
  Astrophys.\ J.\  {\bf 383}, 60 (1991).  
  
%\cite{Davis:2003ad}
\bibitem{Davis:2003ad} 
  T.~M.~Davis and C.~H.~Lineweaver,
  %``Expanding confusion: common misconceptions of cosmological horizons and the superluminal expansion of the universe,''
  Proc.\ Astron.\ Soc.\ Austral.\ 
  [Publ.\ Astron.\ Soc.\ Austral.\  {\bf 21}, 97 (2004)]
  doi:10.1071/AS03040
  [astro-ph/0310808].
  %%CITATION = doi:10.1071/AS03040;%%
  %71 citations counted in INSPIRE as of 24 Nov 2017

%\cite{Ellis:2015wdi}
\bibitem{Ellis:2015wdi} 
  G.~F.~R.~Ellis and J.~P.~Uzan,
  %``Causal structures in inflation,''
  Comptes Rendus Physique {\bf 16}, 928 (2015)
  doi:10.1016/j.crhy.2015.07.005
  [arXiv:1612.01084 [gr-qc]].
  %%CITATION = doi:10.1016/j.crhy.2015.07.005;%%
  %2 citations counted in INSPIRE as of 24 Nov 2017
  
%\cite{Margalef-Bentabol:2012kwa}
\bibitem{Margalef-Bentabol:2012kwa} 
  B.~Margalef-Bentabol, J.~Margalef-Bentabol and J.~Cepa,
  %``Evolution of the Cosmological Horizons in a Concordance Universe,''
  JCAP {\bf 1212}, 035 (2012)
  doi:10.1088/1475-7516/2012/12/035
  [arXiv:1302.1609 [astro-ph.CO]].
  %%CITATION = doi:10.1088/1475-7516/2012/12/035;%%
  %2 citations counted in INSPIRE as of 24 Nov 2017

%\cite{MargalefBentabol:2013bh}
\bibitem{MargalefBentabol:2013bh} 
  B.~Margalef-Bentabol, J.~Margalef-Bentabol and J.~Cepa,
  %``Evolution of the Cosmological Horizons in a Concordance Universe with Countably Infinitely Many State Equations,''
  JCAP {\bf 1302}, 015 (2013)
  doi:10.1088/1475-7516/2013/02/015
  [arXiv:1302.2186 [astro-ph.CO]].
  %%CITATION = doi:10.1088/1475-7516/2013/02/015;%%
  %2 citations counted in INSPIRE as of 24 Nov 2017
  
%%%%%%%%%%%%%%%%%%%%%%%%%%%%%%%%%%
%\cite{Martin:2004pm}
\bibitem{Martin:2004pm} 
  J.~Martin and P.~Peter,
  %``On the properties of the transition matrix in bouncing cosmologies,''
  Phys.\ Rev.\ D {\bf 69}, 107301 (2004)
  doi:10.1103/PhysRevD.69.107301
  [hep-th/0403173].
  %%CITATION = doi:10.1103/PhysRevD.69.107301;%%
  %28 citations counted in INSPIRE as of 23 Nov 2017

%\cite{Brandenberger:2012zb}
\bibitem{Brandenberger:2012zb} 
  R.~H.~Brandenberger,
  %``The Matter Bounce Alternative to Inflationary Cosmology,''
  arXiv:1206.4196 [astro-ph.CO].
  %%CITATION = ARXIV:1206.4196;%%
  %126 citations counted in INSPIRE as of 23 Nov 2017

%\cite{Cai:2012va}
\bibitem{Cai:2012va} 
  Y.~F.~Cai, D.~A.~Easson and R.~Brandenberger,
  %``Towards a Nonsingular Bouncing Cosmology,''
  JCAP {\bf 1208}, 020 (2012)
  doi:10.1088/1475-7516/2012/08/020
  [arXiv:1206.2382 [hep-th]].
  %%CITATION = doi:10.1088/1475-7516/2012/08/020;%%
  %178 citations counted in INSPIRE as of 23 Nov 2017 

%\cite{Martin:2003bp}
\bibitem{Martin:2003bp} 
  J.~Martin and P.~Peter,
  %``On the causality argument in bouncing cosmologies,''
  Phys.\ Rev.\ Lett.\  {\bf 92}, 061301 (2004)
  doi:10.1103/PhysRevLett.92.061301
  [astro-ph/0312488].
  %%CITATION = doi:10.1103/PhysRevLett.92.061301;%%
  %57 citations counted in INSPIRE as of 23 Nov 2017  
%%%%%%%%%%%%%%%%%%%%%%%%%%%%%%%%%%
  
%\cite{Rubakov:2014jja}
\bibitem{Rubakov:2014jja} 
  V.~A.~Rubakov,
  %``The Null Energy Condition and its violation,''
  Phys.\ Usp.\  {\bf 57}, 128 (2014)
  [Usp.\ Fiz.\ Nauk {\bf 184}, no. 2, 137 (2014)]
  doi:10.3367/UFNe.0184.201402b.0137
  [arXiv:1401.4024 [hep-th]].
  %%CITATION = doi:10.3367/UFNe.0184.201402b.0137;%%
  %67 citations counted in INSPIRE as of 23 Nov 2017

%\cite{Paul:2014cxa}
\bibitem{Paul:2014cxa} 
  N.~Paul, S.~N.~Chakrabarty and K.~Bhattacharya,
  %``Cosmological bounces in spatially flat FRW spacetimes in metric $f(R)$ gravity,''
  JCAP {\bf 1410}, no. 10, 009 (2014)
  doi:10.1088/1475-7516/2014/10/009
  [arXiv:1405.0139 [gr-qc]].
  %%CITATION = doi:10.1088/1475-7516/2014/10/009;%%
  %9 citations counted in INSPIRE as of 23 Nov 2017

%\cite{Bhattacharya:2015nda}
\bibitem{Bhattacharya:2015nda} 
  K.~Bhattacharya and S.~Chakrabarty,
  %``Intricacies of Cosmological bounce in polynomial metric $f(R)$ gravity for flat FLRW spacetime,''
  JCAP {\bf 1602}, no. 02, 030 (2016)
  doi:10.1088/1475-7516/2016/02/030
  [arXiv:1509.01835 [gr-qc]].
  %%CITATION = doi:10.1088/1475-7516/2016/02/030;%%
  %4 citations counted in INSPIRE as of 23 Nov 2017  

\bibitem{poisson}
  E.~Poisson, ``An advanced course in general relativity,''
  https://www.physics.uoguelph.ca/poisson/research/agr.pdf

\bibitem{grad}  
I.~S. Gradshteyn and I.~M. Ryzhik, ``Table of Integrals, Series, and Products,''
Seventh Edition (Edited by A. Jeffrey and D. Zwillinger), Page 73.  

%\cite{Barrau:2017ukm}
\bibitem{Barrau:2017ukm} 
  A.~Barrau, K.~Martineau and F.~Moulin,
  %``Seeing through the cosmological bounce: Footprints of the contracting phase and luminosity distance in bouncing models,''
  arXiv:1711.05301 [gr-qc].
  %%CITATION = ARXIV:1711.05301;%%


%%%%%%%%%%%%%%%%%%%%%%
\end{thebibliography}
\end{document}
